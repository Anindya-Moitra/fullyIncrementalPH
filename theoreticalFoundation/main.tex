\documentclass[11pt]{article}

\usepackage{fullpage}
\usepackage{times}
\usepackage{cite}
\usepackage{graphicx}
% \usepackage{algorithm}
% \usepackage{algpseudocode}
\usepackage{amssymb}
\usepackage{siunitx}
\usepackage{caption}
\usepackage{url}
\usepackage{hyperref}
\usepackage{amsfonts}
%\usepackage{amsmath}
\usepackage{csquotes}
\usepackage{url}
\usepackage{amsthm}
\usepackage{listings}
\newtheorem{theorem}{Theorem}[section]
\usepackage{color}
\usepackage{verbatim}
\usepackage{mathtools}
\usepackage{relsize}


\begin{document}

\title{Streaming Version of the Matrix Reduction}

\maketitle

\section{The Problem}

This document addresses one of the primary challenges of computing persistent homology on streaming 
data by a fully incremental approach.  As a data stream is a potentially infinite sequence of data 
objects, the entire stream cannot be stored in the memory typically available to a computer.  
Therefore, the computation of persistent homology on streaming data requires an incremental 
approach.  A couple of computational models for applying persistent homology on data streams are 
being developed as two separate projects that involve partially incremental approaches.  In 
particular, consistent with the standard computational paradigm \cite{silva-13} for processing 
data streams, each of those two models consists of two principal components: (i) \emph{online}, and 
(ii) \emph{offline}.  However, yet another (\emph{i.e.}, a third) computational model can be 
developed by a \emph{fully online} or \emph{fully incremental} approach.


A key requirement for developing such a fully incremental model for persistent homology would be 
the ability to perform the \emph{Gaussian elimination} (also called the \emph{reduction}) of the 
boundary matrix \cite{edelsbrunner-00, zomorodian-05} by an incremental algorithm.  The Gaussian 
elimination is performed during the offline component (\emph{i.e.}, as a \emph{batch processing} 
mechanism) in the previous two models for computing persistent homology on streaming data.


 

 


This document attempts to develop 
the theoretical foundation for performing the \emph{Gaussian elimination} (also called the 
\emph{reduction}) of the boundary matrix \cite{edelsbrunner-00, zomorodian-05} by an incremental 
algorithm.  The Gaussian elimination of the boundary matrix is performed during the offline 
component (as a \emph{batch processing} mechanism) in the previous two models for computing 
persistent homology on streaming data.


The standard algorithm \cite{edelsbrunner-00, zomorodian-05} \emph{reduces} the boundary matrix 
$\partial$ of a filtration \cite{zomorodian-10} to the column-echelon form $R$.  Usually, the 
entire boundary matrix or a simplified data structure thereof is processed in the memory while the 
standard algorithm is executed.  This approach is not desirable when computing persistent homology 
on streaming data.  When working with data streams, one would want to add a simplex $\sigma$ to the 
already reduced matrix $R$ without having to recompute the reduction of other columns due to the 
addition of $\sigma$.  Ideally, only the column of $\sigma$ should be reduced as the simplex 
$\sigma$ is added to the filtration.



\bibliographystyle{IEEEtran}
\bibliography{refs}

\end{document}