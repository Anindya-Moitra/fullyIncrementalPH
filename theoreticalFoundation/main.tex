\documentclass[11pt]{article}

\usepackage{fullpage}
\usepackage{times}
\usepackage{cite}
\usepackage{graphicx}
% \usepackage{algorithm}
% \usepackage{algpseudocode}
\usepackage{amssymb}
\usepackage{siunitx}
\usepackage{caption}
\usepackage{url}
\usepackage{hyperref}
\usepackage{amsfonts}
%\usepackage{amsmath}
\usepackage{csquotes}
\usepackage{url}
\usepackage{amsthm}
\usepackage{listings}
\newtheorem{theorem}{Theorem}[section]
\usepackage{color}
\usepackage{verbatim}
\usepackage{mathtools}
\usepackage{relsize}


\begin{document}

\title{Streaming Version of the Matrix Reduction}

\maketitle

\section{The Problem}

The standard algorithm \cite{edelsbrunner-00, zomorodian-05} \emph{reduces} the boundary matrix 
$\partial$ of a filtration \cite{zomorodian-10} to the column-echelon form $R$.  Usually, the 
entire boundary matrix or a simplified data structure thereof is processed in the memory while the 
standard algorithm is executed.  This approach is not desirable when computing persistent homology 
on streaming data.  As a data stream is a potentially infinite sequence of data objects, the entire 
stream cannot be stored in the memory typically available to a computer.  Therefore, the 
computation of persistent homology on streaming data requires an incremental approach.  When 
working with data streams, one would want to add a simplex $\sigma$ to the already reduced matrix 
$R$ without having to recompute the reduction of other columns due to the addition of $\sigma$.  
Ideally, only the column of $\sigma$ should be reduced as the simplex $\sigma$ is added to the 
filtration.

\bibliographystyle{IEEEtran}
\bibliography{refs}

\end{document}